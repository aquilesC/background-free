\documentclass[twocolumn]{article}
\usepackage{graphicx}
\usepackage{amsmath}
\usepackage{amssymb}
\usepackage{subfigure}
\usepackage{epstopdf}
\epstopdfsetup{update} % only regenerate pdf files when eps file is newer
\title{Background-Free imaging of gold nanorods in live cells}
\author{Aquiles Carattino \and \and Veer Keizer \and Michel Orrit}

\begin{document}
\maketitle
\abstract{This is the abstract}
\section{Introduction}
In recent years metallic nanoparticles received a fair amount of attention
because of their optical properties. The first to mention is their great
stability, specially when compared to fluorescent dyes: nanoparticles do not
blink nor bleach, making them ideal for labeling tracking biological samples
over extended periods of time. 

Another interesting property of gold nanoparticles is that they exhibit a
collective oscillation of surface electrons called the plasmon. This oscillation
will exhibit a resonance that is strongly dependent on the shape of the
particles. For instance spheres will have a resonance at around $540\,nm$ and if
they get elongated (i.e. they become rods) the resonance can be shifted down to
over $800\,\nm$. It is important to note that this tuning can be easily achieved
at the moment of the synthesis with a narrow distribution of sizes. 

The plasmon is responsible for an enhanced cross section of the particle in a
specific wavelength range but it is also responsible for an enhanced
emission in the same region. This overlap between the excitation spectra and the
emission spectra of nanoparticles makes them behave in a complete different way
than molecules in which a Stokes-shift is easy to identify and exploit. 

The luminescence of rods, for instance, can be exited off-resonance with a
short-wavelength laser (for instance at $532\,\nm$) and the emission will be
mostly concentrated close to the plasmon resonance. This is the closer picture
to a single-molecule experiment, in which the Stokes-shifted emission can be
easily detected by using a long pass filter. The drawback, however, is that the
cross-section of the particle is much smaller at this wavelengths. The other
alternative is to excite at the plasmon resonance which will yield a much
stronger absorption but at the expense of filtering out much of the emission,
since the Stokes-shift is relatively small. 

This works focuses in the exploitation of another kind of emission that is
surprisingly efficient in gold nanoparticles, the anti-Sokes luminescence. In
this case when exciting in resonance with the plasmon, a short pass filter can
be introduced for observing only the emission at higher energies. The main
advantage of this procedure is that it allows to highly reduce the background
arising from self-fluorescence and Raman-scattering (anti-Stokes Raman will also
be present but to a much smaller extent.) 

Other techniques normally employed in microscopy for achieving a lower
background include two-photon excitation, photothermal detection or heavy image analysis.
The first implies access to a pulsed laser, the second needs the
correct alignment ad overlap of lasers with different wavelengths and the latter
normally requires the proper choosing of dyes (and doesn't work with
nanoparticles since they do not blink.) Exploiting the anti-Sokes luminescence
only needs a suitable shortpass filter, hence can readily be achieved in any
normal confocal or wide-field microscope. 

The mechanism for explaining this emission is not fully understood as for now.
Different papers propose alternative explanations that can be summarized as
follows: an absorbed photon generates an electron-hole pair with an energy
difference equivalent to the energy of the excitation. After an interaction with
the gold lattice (i.e. with a phonon) either the hole gives energy to it or an
electron receives energy from it. This will increase the gap between electron
and hole thus exhibiting an emission with a shorter wavelength (higher energy).
A schematic of this process can be seen in Figure \ref{fig:anti-Stokes-process},
where the Stokes-shifted emission is also explained via a relaxation of the
electron-hole pair through interactions with the lattice. 

Even if the mechanism is subject to a fair amount of debate, it doesn't hinter
the possibility of exploiting it for imaging purposes. It was shown before that
the quantum yield of the anti-Stokes emission is smaller than the one of its
counterpart the Stokes-shifted, but the reduction of background for equivalent
excitation powers makes the signal-to-background ratio much higher and allows
the detection of single nanoparticles even in high-background situations as may
be a stained cell. 

\section{Experimental method}
The core of this work focuses in the employment of a combination of a notch
filter and a  Smerock $633\,\nm$ short pass filter in a home built confocal
microscope. The objective is an oil-immersion Olympus 60X NA 1.4 that allows a
great efficiency in exciting the particles and collecting their emission. An APD
is employed the detection and a spectrometer Acton 500i for measuring the
emission spectra of the particles.

The samples are prepared by spin coating a suspension of gold nanorods into a
clean coverslip as described elsewhere. The employed rods have a nominal size of
$25\,\nm$x$60\,\nm$ hence the plasmon resonance roughly coincides with a HeNe
laser with wavelength $633\,\nm$. A second laser with wavelength $532\,\nm$ is
employed for exciting the luminescence of the particles and collecting the full
plasmonic spectrum and corroborate that they are single particles. 

For proving that the technique is well suited for imaging of particles in
biological conditions, U2OS cells where deposited on top of the samples. This
allows to characterize the emission from the particles avoiding their diffusion
(acquiring spectra takes several seconds). It may be argued that this condition
is not exactly imaging of rods inside a cell, but resembles the condition of
imaging in cell membranes without going into the trouble of the
functionalization of the particles. For tracking of diffusing particles inside
cells going to a wide-field setup is convenient and we show that is possible to
achieve with the anti-Stokes imaging. 

\section{Results and discussion}
The firs step in the experiments is to characterize the luminescence emission of
gold nanorods. For this a suspension of particles is spincasted on a clean
coverslip, assuring that there is no formation of clusters and that individual
particles can be resolved. Figure \ref{fig:spectra_rod} shows the typical
spectra of a gold nanorod while excited with $532nm$ (green curve). In this case
it can be clearly seen the full spectrum of the longitudinal plasmon with it's
peak at $620\,\nm$. If the particle is excited at, or close to, the resonance,
the shape of the spectrum changes drastically: the red curve shows the Stokes
shifted emission while exciting at $633\,\nm$ (the vertical black line). It was
normalized to show the overlap with the curve obtained with the $532\,\nm$
laser. The blue curve shows the anti-Stokes emission at shorter wavelengths
(higher energies) and it can be observed that the shape (even with the same
normalization constant as for the Stokes case) doesn't resemble at all to the
plasmon resonance. 

The dashed curves in Figure \ref{fig:spectra_rod} show the non normalized
emission from the nanoparticle. This allows to have an insight of the obtainable
signal with an excitation power in the back frocal plane of the objective of
$100\,\mu W$. It has to be noted that the anit-Stokes emission is greatly
enhanced by the plasmon position, therefore exciting in the red wing of the
plasmons gives the highest blue-shifted emission and the opposite for the
red-shited one. This means that this particular particle shows one of the best
attainable ratios anti-Stokes/Stokes emission. 

From here on we are focusing in the anti-Stokes emission, since this will
provide the best background rejection. The inset of the same Figure shows the
linear behavior of this emission intensity while varying the excitation
intensity, assuring its one photon nature. The same is observed regardless of
the plasmon peak position and for both the Stokes and the anti-Stokes sides.

Polarization analysis of the emission show that both emissions happen along the
longitudinal axis of the particle. This is a great indication that the emission
is of a plasmonic nature, as was argued in the introduction and was pointed out
in different works. Figure \ref{fig:emission_peak_position} shows the dependence
of the emission intensity as a function of the plasmon peak position for both
emissions. The inhomogeinity observed for both emissions can be attributed to a
distribution of sizes of the sample (bigger particles will have a bigger cross
section). It is possible to observe, however, that the maximum emission
intensity for the anti-Stokes appears for those particles with the plasmon
slightly blue-shifted from the laser wavelength, while the opposite is observed
for the Stokes emission. This means that there is a trade off between the
excitation efficiency and the collection efficiency: exciting in perfect
resonance is more efficient, but the filters will eliminate the most significant
part of the emission. Exciting off-resonance even if with a smaller
cross-section, improves the collection efficiency. Therefore the particle
described earlier, in Figure \ref{fig:spectra_rod} fulfills this condition for
maximum efficiency. 

Figure \ref{fig:shortpass_longpass} shows two raster scans of the same
$10x10\,\mu m^2$ of the sample under the same irradiation conditions; the first
one shows the Stokes emission and the other the anti-Stokes. It can readily be
seen that the same particles are observed with both sets of filters.
As explained earlier a bright particle with the long pass me be much dimer with
the short pass and vice versa because of a different plasmon peak position. It
is important to note the difference in background between both techniques. For
the anti-Stokes a lot of bright pixels are present and entire regions show a
much higher count rate, coming from different parts of the cell above the
observed area. In the case of the anti-Stokes, this bursts disappear but still
there are some regions with higher count rates than the average background.  We
attribute this to anti-Stokes Raman arising from organelles and is a proper
feature of cells; repeating the same experiment with the rods immersed only in
water does not show any background above dark counts from the detector.

It is important for comparing the results obtained with the Stokes and the
anti-Stokes emission to study the dependence of the signal and the background as
a function of excitation power. Figure \ref{fig:power_intensity} shows exactly
this for a particle which plasmon peak position was on top of the laser
wavelength (not favouring either kind of luminescence). For the Stokes side
(red curves) a steady increase in signal is accompanied by an increase in
background, hence the ratio saturates even for low powers. The anti-Stokes
(blue-curves) however shows a much steeper increase of the signal as compared
to the background. This sadly cannot be extrapolated to even higher excitation
intensities because would start reshaping the particles. 

\section{Conclusions}
In this work we have proved that the anti-Sokes emission arising from the
excitation in (or close to) the plasmon resonance of a gold nanorod can be
exploited to image them in biological conditions. The comparison between the
Stokes and anti-Stokes was possible by using immobilized particles, but this
proves that the same technique can be employed for imaging particles in fixed
cells, for example. 

Extending this technique to wide-field should be possible considering the powers
employed in this work. EMCCDs should provide enough gain as to easily detect
single nanoparticles, while at the same time the background is considerable low
(or negligible for low powers). This extention would provide a way of doing
tracking at higher framerates than achivable by confocal scanning. 

There is however a drawback in the techique that is the lower countrate as
compared to the Stokes emission. For doing localization it is usually better to
have a higher countrate than a better signal-to-background ratio. On the other
hand this work shows the possibility of observing particles in less favorable
conditions, as may be a stained cell, or smaller particles which will have
smaller cross sections but still be brighter than the background. 

It is important to remember that this anti-Stokes emission will be present also
in two-photon experiments, and depending on the set of filters employed it may
not be negligible. There are many paths to explore


\end{document}
