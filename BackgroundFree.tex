\documentclass[twocolumn]{article}
\usepackage{graphicx}
\usepackage{amsmath}
\usepackage{amssymb}
\usepackage{subfigure}
\usepackage{epstopdf}
\epstopdfsetup{update} % only regenerate pdf files when eps file is newer
\title{Background-Free imaging of gold nanorods in live cells}
\author{Aquiles Carattino \and Michel Orrit}

\begin{document}
\maketitle
\abstract{This is the abstract}
\section{Introduction}
This is an introduction

\section{Experimental method}
A normal confocal setup is employed; it is possible to change from $532\,\nm$ to
$633\,nm$ excitation laser

\section{Results and discussion}
The first set of experiments are aimed to characterize the luminescence emission
of gold nanorods. For this a suspension of particles is spincasted on a clean
coverslip, assuring that there is no formation of clusters and that individual
particles can be resolved. Figure \ref{fig:spectra_rod} shows the typical
spectra of a gold nanorod while excited with $532nm$ (green curve) and with
$633nm$ (red curve); the spectra was normalized at $650nm$ to show the overlap
between both. For the latter, the emission with wavelength shorter than the
excitation wavelength of the laser corresponds to the anti-Stokes emission and is the
portion that will be used for the single-particle detection with high background
rejection. The inset shows the linear behavior of this emission intensity while
varying the excitation intensity, assuring its one photon nature.

By introducing a $633nm$ short-pass filter (Semrock Razor Edge) in a normal
confocal setup it is possible to observe only the anti-stokes part of the
emission spectrum, rejecting entirely any background that may arise from the
self fluorescence of the cell. As a first approach We used a sample prepared as
described before with cells deposited on top. This was done in order to have
immobilized particles, allowing to compare the difference between the signal
obtained from the standard configuration with a long-pass (LP) filter and with a
short-pass (SP) filter and to obtain the spectra of the particles. 

Figure \ref{fig:rod-in-glass-cell} shows the results obtained in such
conditions. The image obtained with the shortpass
(\ref{fig:rod-in-glass-cell}.A) shows that the background disappears entirely;
comparing with \ref{fig:rod-in-glass-cell}.B, that was taken with a longpass
filter, already shows the significant improve of this technique. The different
intensities for different rods arises from different plasmon peak positions. The
insets in both figures show line cuts through the same particle; it can readily
be seen that in the case of the shortpass filter, the background is close to 
dark count rate of the detector. 

%\ref{fig:rod-in-cell}.A shows the confocal scan of an area of $10\,\mu m \times
%#10\,\mu m$ using the SP filter, while in \ref{fig:rod-in-cell}.B the same area
%is observed using a long-pass filter. Each bright spot corresponds to a single
%#rod, as could be determined by acquiring the spectra. Figure
%#\ref{fig:rod-in-cell}.C shows white light transmission image; the square marks
%#the position where the scan was done. Figure \ref{fig:rod-in-cell}.D shows
%line-cuts across one particle; the dashed line being the result corresponding
%#to the LP filter while the full line by using the SP. 

From this Figures it can readily be seen that the signal-to-background while
imaging with the short-pass filter for a fixed intensity is limited by
the dark counts of the detector (in this case $~150cps$) and the values can be
confirmed by taking timetraces of the APD on top of the particle and on the
background.





\section{Conclusions}


\end{document}
