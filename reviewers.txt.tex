We appreciate the kind words of the three reviewers. Below, we answer
individually the raised concerns and comments, pointing out the modifications
included in the text. The three reviewers pointed out concerns regarding the
heating of the nanoparticles as they could damage the cells. We have answered
each individual point and given a more general perspective in the
answer to Reviewer #3. 

We have marked in bold all the modifications included in the text, and tried to
explicitly state in our answers where those modifications were done.

 
Peer Reviewer #1:
 
In this paper the authors describe the use of anti-Stokes emission of imaging
gold nanorods even in the presence of a high fluorescence background. This is an
interesting technical advance that will be viewed with interest by some readers.
There are a few points of clarification and additional experiments that are
required before the paper is ready for publication:
 
1) The authors should more clearly define what they mean by luminescence, and
explain how it differs from scattering in the specific case of gold particles. 

To make a clear distinction between scattering and luminescence we have modified
the 5th paragraph of the introduction; now it clearly states what we refer to
when mentioning the luminescence. Luminescence arises from the radiative
recombination of electron and holes in the metal; it is possible however that
the emission has exactly the same energy than the excitation and therefore would
be indistinguishable from elastic scattering. However the filters needed to
prevent excitation light from reaching the detectors block this portion of the
spectrum.
 
2) The authors should show the absorption spectra of the gold nanorods used
here.

The extinction spectrum of the rods used in this work is presented in the
Supplementary material, Figure S2, together with a TEM image of the particles,
in Figure S4.
 
3) If possible, the authors should estimate the approximate quantum yield for
anti-Stokes luminescence for the particles used here. I was curious to know it,
at least to an order of magnitude, and I'm sure other readers would like to know
this too.

To answer this point, we have included a paragraph in the text with a short
discussion regarding the quantum yield of the anti-Stokes emission. It has to be
noted that this paragraph is slightly speculative, as we didn't perform
single-particle absorption characterization and the reported quantum yields of
particles normally rely on exciting off resonance. See for instance the work of
Yorulmaz et al. Nano Lett. 2012, 12, 4385−4391.
 
4) Figure 3 was hard for me to interpret, since it describes Stokes and
anti-Stokes, but has emission both above and below the excitation wavelength in
both cases. This does not accord with my understanding of Stokes and
anti-Stokes, which is defined relative to the absorption or loss of vibrational
quanta rather than by the wavelength of the emission relative to the absorption
maximum. It's quite possible that my understanding is incorrect, but in any case
the authors should clearly define what they mean here, and clarify this point
for readers like myself. 

Figure 3 represents the total anti-Stokes (a) and Stokes (b) emissions exactly
as defined by the reviewer; the horizontal axis is the position of the resonance
of each particle and not the energy of the emission. To make it clearer, we
have modified the caption of the figure and included a comment in the main text.
It now reads:
``The anti-Stokes emission was calculated by integrating
the recorded spectra at wavelengths shorter than the excitation laser, while the
opposite was done for the Stokes. This can also be directly measured by
recording the emission intensity after a short pass or long pass filter.''
 
5) Does the illumination intensity used here cause local heating? 

The illumination indeed causes local heating. The same concern was raised by the
three reviewers and is now addressed in the main text in the last paragraphs
of the discussion. Moreover a viability test was performed in order to ensure
that even after irradiating the nanoparticles in the vicinity of the cells they
show no damage. Please refer to the answer to Reviewer #3 for further
clarification.
 
6) The authors should measure quite a few more particles (n>10) for figure 6. 2
is not at all sufficient given the variability in particle properties (see fig.
3).

To address this concern, we have added a new Figure to the supplementary
material (Fig. S8). And we mention it in the main text when discussing figure 6.
We show the signal-to-background ratio for more than 10 particles in each case:
Stokes and anti-Stokes under non-stained cells and anti-Stokes under stained
cells.
 
7) The experiment using ATTO 647N to provide background fluorescence is nice
(Fig. 5 and 6), but is not analyzed in sufficient detail. The authors should
provide an image of the labeled cells so that the reader can see if the dye is
uniformly labeling the membrane. Also, the authors should more clearly define
the dark counts, background counts in the presence of the cell but no dye, and
counts resulting from the dye for both the Stokes and anti-Stokes measurement.

Part of these remarks were already present in the main text but maybe not
sufficiently clear. We have therefore updated the text according to the
suggestions of the reviewer, we now explicitly wrote the count rate of the
background in the presence of cells with and without dye for both measurements. 

Figure S7 shows a 20X20um^2 scan of the stained cells where it is possible to
observe the background signal. As stated in the text, we have incubated the
cells with ATTO 647n, therefore staining not only the cell membrane but also the
cytoplasm. Since this was a non-specific staining, we couldn't prevent the
formation of regions with higher densities of dye and this appears as the
places where a higher count rate is observed (for instance in the top left
part of the image and in the bottom left.)
 
8) No imaging technique that I am aware of is literally background free. The
title should be changed in this regard.
 
Peer Reviewer #2:
 
Carratino et al. describe the use of anti-Stokes detection of nanorod emission
for background free imaging even in the presence of alternative fluorescence
sources. The technique is potentially useful for multimodal imaging, enabling
for example single particle tracking together with larger field of view imaging.
He approach is demonstrated with a confocal microscope, but should be usable
with any fluorescence microscope.
 
Impressively, they show that even in the presence of cells containing
fluorescent dyes excited by the same illumination wavelength, they can still
clearly identify the nanorod emitters. The work is interesting, novel and
potentially useful for a wide audience and I therefore recommend publication,
after minor revisions suggested below.
 
Major Comments:
 
1. The abstract states that: "we show that even in a cell containing the
fluorescent dye ATTO 647N, the signal-to-background ratio of the anti-Stokes
emission is higher than 10" and in the conclusion "this work shows that the
technique can be easily extended to imaging in fixed cells, in vivo or even for
tracking particles in real time"; but as they state themselves, no imaging
inside a cell or even technically in the cell membrane is presented. The wording
should thus be adjusted to more accurately reflect what actually has been done
and what can be potentially done in the future. 

We agree with the comments of the reviewer; the abstract and the conclusions
have been changed to reflect this.
 
2. The technique relies on illuminating the gold nanorods on resonance but makes
no mention of local heating as a result of absorption that could be a problem
for imaging in cells. It is stated that one cannot illuminate above 53 kW/cm2
because the rods reshape, and that the luminescence quantum yield is on the
order of 10-6, but no mention is made as to what happens for illumination powers
on the order of 15-30 kW/cm2, which is the range that seems to be predominantly
used. If 53 kW/cm^2 causes effectively melting of a metallic rod, I am concerned
what would happen to cellular material at only half that illumination power.
This should be clarified to ensure that sufficient Anti-Stokes photon flux is
achievable even in a realistic imaging environment. 

Heating of particles is indeed a major concern for imaging purposes. Firstly we
would like to point out that the reshaping of the particles happens at lower
temperatures than melting of gold. The first is related to a rearrangement of
gold atoms from the surface, leading to sphere-like particles and can happen at
temperatures as low as 100 degrees C; the latter depends on the phase transition
of gold at around 1000 degrees C. This difference, for instance, allows to
excite gold spheres with much higher powers than nanorods. For further
clarifications in this point, please refer to the answer the Reviewer #3.


Minor Comments/Clarifications:
 
1. The labelling of figures is confusing in that reference to figures in the
main text and supplementary information is not clearly separated. On page 3
reference to Figure 1 implies the first figure in the main text, while on page 6
Figure 1b refers to the first figure in the supplementary/methods section. 

This has been corrected. Figures in the supplementary text are now labelled with
a preceding S.
 
2. In the results section at the top of page 10, the sentence "In both cases the
irradiation intensity was kept at 30 kW/cm2 in the back aperture of the
objective" is unclear to me. I assume they mean that the intensity was 30 kW/cm2
at the sample, because 30 kW/cm2 in the back aperture being focused to a spot
on the would be an enormous amount of power. 

Indeed, it was a mistake in the text that has now been corrected. The power was
30kW/cm2 at the sample.
 
3. It would have been nice to see what the images/contrast look like when imaged
onto a camera, to see how the signal-to-background ratio looks without the help
of confocal imaging, as widefield imaging is mentioned as a potential
alternative. 

We agree that it would have been a nice addition to the paper to show a
widefield image. However at the moment of the experiments no camera was
available. We may work on this for a later experiment.

 
Peer Reviewer #3:
 
In the manuscript "Background-free imaging of gold nanorods through detection of
anti-Stokes emission" the authors describe an elegant and simple method to image
gold nanorods within cells. This novel contrast agent has many unique and useful
features compared to dyes.
 
The physical part of the work is rigorous and well described.
 
It seems necessary that for the intended biological applications, the authors
address the issue of damage to the cells with some more detail.
 
However, if this is concincingly answered, the work is a *very* important
breakthrough in cell imaging. Having a new class of contrast agents will lead to
novel insights in many fields. I would therefore strongly support timely
publication of this work in this journal.
 
Detailed comments:
 
1. The laser power densities used in this work (in the order of kW/cm2) could
potentially lead to high temperatures if the nanoparticles - as sometimes used
in Plasmonic Photothermal Therapy (PPTT). In addtion, photogenerated singlet
oxygen species photodynamic therapy) could be an issue.
 
Ideally, the authors would either show cell viability after imaging by some
standart test (for example a simple alamar blue test)- or show convincing
theoretical arguments why those problems should not arrise.

Indeed the concern about the temperatures reached by the particles has been
raised by the all the reviewers. To address this questions we have included two
paragraphs at the end of the discussion in which we argue that the dissipated
power by a single nanoparticle is several orders of magnitude lower than the
reported for Photothermal Therapy. See for instance 

Huang, X., Jain, P. K., El-sayed, I. H. & El-sayed, M. A. Determination of
the Minimum Temperature Required for Selective Photothermal Destruction of
Cancer Cells with the Use of lmmunotargeted Gold Nanoparticles. Photochem.
Photobiol. 412–417 (2006). doi:10.1562/2005-12-14-RA-754

Moreover it is possible to calculate the temperature increase of the particle
under standard irradiation conditions and see that it is of 25 degrees, while in
the same paper of Huang et Al. they mention temperatures of 70 degrees to induce
cell death. Moreover the temperature increase in our case is limited to the
vicinity of one particle, while in the other work is a consistent increase in
temperature across the cell. 

Finally we have performed a viability test of the cells after being irradiated
with the laser and see no difference with those which were not irradiated. This
figure was added to the supplementary material and is referred to in the
results of the main text.
 
2. Did you observe NP uptake?

Because of how the samples are prepared, it is not possible to observe NP
uptake. Gold nanorods are fixed to the glass substrate to avoid their diffusion
out of focus. A more detailed explanation on the sample preparation can be
found, for instance in:
 Zijlstra, P. & Orrit, M. Single metal nanoparticles:
optical detection, spectroscopy and applications. Reports Prog. Phys. 74, 106401
(2011).
 
3. Could you show overview pictures of the cells with lower magnification (e.g.
20x or 40x) to compare irradiated and non-irradiated areas?

These images have been included in the supplementary material, together with
viability tests. There is no visible difference between the irradiated and
non-irradiated regions.
 
4. Please provide details on laser spot size, laser fluence and type (pulsed, cw
laser) prominently in the main text 

In this work we employed two CW lasers (532nm and 633nm). We have now explicitly
stated in the experimental section that they are continuous and not pulsed. Both
lasers used were focused to a diffraction-limited spot through a high-NA
objective. This was now clearly stated in the text.
